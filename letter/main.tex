%
% Copyright(C) Pedro Henrique Penna <pedrohenriquepenna@gmail.com>
%
% All rights reserved.
%

\documentclass[a4paper,10pt,english]{article}
\usepackage[paper=a4paper,centering,margin=1in]{geometry}

% Font
\usepackage{palatino}

% Encoding
\usepackage[utf8]{inputenc}
\usepackage[T1]{fontenc}

% Language
\usepackage[english]{babel}

% Acronyms
\usepackage{xspace}
\usepackage[acronym,nowarn]{glossaries}
\glsdisablehyper
%
% Copyright(C) Pedro Henrique Penna <pedrohenriquepenna@gmail.com>
%
% All rights reserved.
%

\newacronym{cart}{CArT}{Computer Architecture and Parallel Processing Team}
	\newcommand{\cart}{\gls{cart}\xspace}

\newacronym{dcc}{DCC}{Departamento de Ciência da Computação}
	\newcommand{\dcc}{\gls{dcc}\xspace}

\newacronym{ppginf}{PPGInf}{Programa de Pós-Graduação em Informática}
	\newcommand{\ppginf}{\gls{ppginf}\xspace}

\newacronym{pucminas}{PUC Minas}{Pontifícia Universidade Católica de Minas Gerais}
	\newcommand{\pucminas}{\gls{pucminas}\xspace}

\makeglossaries

% Hyper References
\usepackage[hidelinks]{hyperref}

% Images
\usepackage{graphicx}
\usepackage{subcaption}
\graphicspath{{./img/}}
\DeclareGraphicsExtensions{.pdf,.jpeg,.png}

% Tables
\usepackage{multirow}

% Paragraphs
\setlength\parindent{0.0em}
\setlength\parskip{0.5em}
\renewcommand{\baselinestretch}{1.15}

%===============================================================================

% Writer Info
\def\firstname{Pedro Henrique\xspace}
\def\lastname{Penna\xspace}
\def\status{PhD Candidate\xspace}

% Institution Info
\def\university{\pucminas}
\def\institute{\dcc}
\def\department{\ppginf}
\def\laboratory{\cart}
\def\city{Belo Horizonte, Brazil\xspace}

\def\prevarticle{%
	``\textit{%
		BinLPT: A Novel Workload-Aware Loop
		Scheduler for Irregular Parallel Loops%
	}''\xspace
}

\def\article{%
	``\textit{%
		A Comprehensive Performance Evaluation of
		the BinLPT Workload-Aware Loop Scheduler%
	}''\xspace
}

\def\journal{Concurrency and Computation: Practice and Experience\xspace}

\def\conference{%
	\textit{
		XVIII Simpósio em Sistemas Computacionais de Alto Desempenho (WSCAD-SSC)
	}\xspace
}

%===============================================================================

\begin{document}


\begin{tabular}{ p{5em} l }
	\multirow{3}{*}{\includegraphics[width=5em]{logos/pucminas.png}}
		& {\large\bf{\university}} \\[0.2em]
		& \institute               \\
		& \department              \\[0.4em]
\end{tabular}

\rule{\linewidth}{0.5pt}

%===============================================================================

% Turn off page numbers.
\pagenumbering{gobble}

\begin{flushleft}
\firstname \lastname, \status\\
\laboratory
\end{flushleft}

\begin{flushright}
\city, \today
\end{flushright}

Dear Editor, \\

I am writing to submit our paper entitled \article for consideration for
publication in \journal, special issue for the \conference.  This work
is an extended version of the paper \prevarticle. We believe that it
includes enough new contributions to justify the journal publication.

In our original work, we proposed a new workload-aware scheduling strategy
called BinLPT to overcome drawbacks of existing related strategies. BinLPT
strategy relies on three features to deliver superior performance:  (i)
user-supplied estimations of the workload of the loop; (ii) a greedy heuristic
that adaptively partitions the iteration space in several chunks; and (iii) a
scheduling scheme based on the LPT and on-demand techniques.  Based on this
original work, we add the following new contributions in this extended version:

	\begin{itemize}
		\item We introduce a multiloop support feature to the original
			implementation of BinLPT. This new functionality enables the HPC
			engineer to reuse workload estimations across different parallel
			loops as well as to have different workload estimations for each
			one. Based on the multiloop feature, we discuss how we were able to
			integrate BinLPT into a large-scale real-world-application: an
			Elastodynamics Simulator. Furthermore, we present a performance
			evaluation of this simulator running on the Santos Dumont
			supercomputer.

		\item We present a comprehensive performance evaluation of BinLPT based
			on three different techniques.  First, we used simulations to
			understand the performance of BinLPT in a great number of
			scenarios.  Next, we relied on a synthetic kernel to uncover the
			upper bound performance gains of our strategy.  Finally, we
			employed three scientific kernels, each of which featuring distinct
			characteristics, to study the performance of BinLPT.  We considered
			several performance factors in our analysis and a wide variety of
			workloads, and we run all experiments with a 192-core NUMA machine
			to provide a rich analysis.
	\end{itemize}

Furthermore, we have re-worked existing sections to better detail the internals
and implementation of BinLPT. Our results unveiled that BinLPT better balances
the workload of a loop, and this behaviour is strengthened by the algorithmic
complexity of the loop.  Overall, BinLPT delivers up to $37.15\%$ and $9.11\%$
better performance than baselines in the application kernels and elastodynamics
simulations, respectively.

Thank you for receiving our manuscript and considering it for review. We
appreciate your time and look forward to your response.\\

Best regards,

\firstname

\end{document}
